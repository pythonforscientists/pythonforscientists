\chapter*{Labs}
\markboth{\MakeUppercase{Labs}}{\MakeUppercase{Labs}}
This chapter contains labs and rubrics for each lab. Your instructor might assign certain labs to you and not others.\par
Any modifications that your instructor makes should take precedence over the lab provided here. You should take care to follow your course's style guide, if your instructor has one. This style guide should give you important information regarding naming conventions, line spacing, whitespace, and other notes like these. Remember to follow your course's style guide!

\section*{Lab 0: Python \& Jupyter Notebooks}
\subsection*{Description}
In this lab, we will be investigating how to use Google Colab, an online tool for creating, editing, and exporting Jupyter Notebooks. A Jupyter Notebook is like a lab notebook for computer scientists and data scientists, and it's a great way to share your code with other computer scientists, since it lets you write code and text inline. Google Colab is just one way of accessing Jupyter Notebooks. You could also use Anaconda, which is a way to edit your Jupyter Notebooks locally.
\subsection*{Task}
You will be guided through this lab step-by-step.\par
\subsubsection*{Lab Preparation}
\begin{enumerate}
	\item In another tab, go to \href{https://colab.research.google.com}{colab.research.google.com}. If this is your first time using Google Colab, you'll probably see a Welcome to Colaboratory document. You can just ignore this. Instead, within your web browser, go to File > New notebook. This will make a new Jupyter Notebook and open it for you. You'll do your lab in this notebook.
	\item Now, at the top of the notebook, you should see that your document is probably named \verb|Untitled0.ipynb|. This is normal, and you can change the name of your notebook by just clicking on the document's name. Go ahead and change it to your username, followed by \verb|Lab 0.ipynb|.
	\item \textit{Hint: Your Jupyter Notebook must always end in .ipynb for it to be recognized as an actual Notebook. ipynb stands for Interactive PYthon NoteBook.}
\end{enumerate}

\subsubsection*{Markdown}
The default first block of your new Jupyter Notebook is a code block. However, we actually don't want to use a code block - we want to use a markdown or text block. So, hover over the existing code block and at the very right side, in the menu that shows up, click on the Trash can icon. This will delete the code block.\par
Now, at the top, underneath the menu bar, you probably see two buttons: one to add code and one to add text.\par
Click on the button to add a new text block.\par
\textbf{i. Text}
\begin{itemize}
	\item When you add text, even if you add Python code, it cannot and will not be executed.
	\item \textbf{Your task}: Go ahead and type in a short introduction for yourself in paragraph form (sentences, not bullet points - we'll learn how to do that later). Include your name, your major, minor or concentrations, and what your favorite dinosaur is.
\end{itemize}
\textbf{ii. Links}
\begin{itemize}
	\item Great! The next most important thing that you'll need to do is put in links. There's only one way to put in links for Jupyter Notebooks:\\
	\begin{lstlisting}[style=none]
[An inline link](//google.com)
	\end{lstlisting}
	\item If you're not sure what to put inside of the brackets, you can always just put the URL in there. That might look like this:\\
	\begin{lstlisting}[style=none]
[An inline link](//google.com)
	\end{lstlisting}
	\item \textit{Hint: Why don't we specify the http: or https: at the beginning? We'll let our web browser figure that out for us. Instead, just put two slashes and the markdown renderer will figure out which protocol to use.}
	\item \textbf{Your task}: Make a new text block. In this text block, make a link that goes to your favorite YouTube video, with the link label stating the name of the video. Put this in a new text block in your lab notebook.
\end{itemize}
\textbf{iii. Lists}
\begin{itemize}
	\item You'll probably also have to make lists. Lists are pretty simple, too. You can create either unordered lists (bullet points) or ordered lists.
	\item Let's start with unordered lists. To create an unordered list, just put a dash and a space at the beginning of each new line with new material.
	\begin{lstlisting}[style=none]
- Item
- Item 
- Item
	\end{lstlisting}
	\item If you want to make an ordered list, instead of a dash, put a number or letter.
	\begin{lstlisting}[style=none]
1. Item 1
2. Item 2
3. Item 3
	\end{lstlisting}
	\item You can also make nested lists. If you are making a nested unordered list, use plus signs for the next level in, then dashes for the next level in, then pluses, and so on and so forth. For each level in, add an indent.
	\begin{lstlisting}[style=none]
- Item
  + Subitem
    - Sub-subitem
      + Sub-sub-subitem
    - Sub-subitem
      + Sub-sub-subitem
      + Sub-sub-subitem
    - Sub-subitem
  + Subitem
  + Subitem
- Item
- Item
	\end{lstlisting}
	\item The same goes for ordered lists.
	\begin{lstlisting}[style=none]
1. Item 1
  1. Item a
    1. Item i
  2. Item b
  3. Item c
2. Item 2
3. Item 3
	\end{lstlisting}
	\item \textbf{Your task}: Now, in your notebook, make a new text block. In this new text block, make an unordered list with the courses you are taking this semester, and make an ordered list of your top five favorite buildings on your campus. If you don't know, just make some buildings up.
\end{itemize}
\textbf{iv. Emphasis}
\begin{itemize}
	\item It's often really handy to emphasize something. Here's how to emphasize stuff in Markdown.
	\begin{lstlisting}[style=none]
Emphasis, aka italics, with *asterisks* or _underscores_.

Strong emphasis, aka bold, with **asterisks** or __underscores__.

Combined emphasis with **asterisks and _underscores_**.

Strikethrough uses two tildes. ~~Scratch this.~~
	\end{lstlisting}
	\item \textbf{Your task}: Now, in your notebook, make a new text block. In this new text block, make some egregious statement like "The earth is flat" and strike it out. Then, make a statement on what your favorite candy is, and bold the candy's name.
\end{itemize}
\textbf{v. Emphasis}
\begin{itemize}
	\item Sometimes, you'll want to put code into your Markdown, like if you want to show what function you're defining or let your reader know what your variable is named.
	\item If you are highlighting code inline (within paragraph text), then you can just use a backquote at the beginning and end of your code chunk. For example {\texttt this} is some code. The backquote character is at the top-left corner of your keyboard, next to the 1/! key.
	\begin{lstlisting}[style=none]
Inline `code` has `back-ticks around` it.
	\end{lstlisting}
	\item Sometimes, you need to list more read-only code than can comfortably fit inside of an inline code mark. Instead, you can write an inline code block. This is still markdown, and it's not executable, but it'll still have Python notation highlighted.
	\begin{lstlisting}[style=none]
```python
s = "Python syntax highlighting"
print s
```
	\end{lstlisting}
	\item This is also known as a fenced code block.
	\item \textbf{Your task}: Now, make a fenced code block that contains the following code, marked down in Python. Add one more line of code, based on something that you've seen in class.
\end{itemize}
\subsubsection*{The Interactive Python Shell}
You can also execute Python code inside of a Jupyter Notebook! This is one of the big advantages of using a Jupyter Notebook instead of in a Word document or Google Doc.\par
Remember that code block that you deleted at the beginning? Now, let's actually use it.\par
In your own lab notebook, make a new code block at the end. Put some Python code that you've run in class, like a \verb|print()| or \verb|input()|. Run it and see what happens! Then, in a new text block below, explain what your code does in using the Markdown skills that you practiced in Part 1.\par
This is the interactive Python shell, and it's great for running small or large chunks of Python code. One of the other cool things about the Python environment is that your variables are maintained throughout your entire notebook. Now, try to define a new variable \verb|students| and assign it some integer value.\par
Now, create a text block explaining where you got the number for \verb|students|.\par
Create a new code block (not the code block that you used above) and print the value of \verb|students|.\par
\subsubsection*{Writing Your Labs in a Jupyter Notebook}
You will write all of your lab writeups in a Jupyter Notebook (one notebook per lab), just like you're doing for this lab. Once you're done. It's not hard to share your finished notebook. At the top, go to File > Download .ipynb. This will download the interactive Python straight to your computer. You can turn this file in. So, export the Jupyter notebook that you've been working on and turn it in using Moodle for the Lab 0 assignment.\par
\textit{Hint: If you export the .py file, all of your markdown will be discarded! Make sure you export as an .ipynb file!}
\subsection*{Turn In}
Turn in your Jupyter Notebook as an \verb|.ipynb| file.

\section*{Lab 1: Error Messages}
\subsection*{Description}
An important part of programming is being able to see how your changes change the behavior of the program. For this exercise, you'll be deliberately making errors and figuring out what lines of code result in these issues.
\subsection*{Task}
Begin with the following code. Confirm that it works correctly.
\begin{lstlisting}[style=pippython]
import pandas as pd
def main():
    ser = pd.Series()
    n_ethereum = int()
    value_per_coin = float()
    total_value = float()
    n_ethereum = int(input("Enter the number of Ethereum in wallet."))
    print('You entered: ' + str(n_ethereum) + '\n')
    
    value_per_coin = float(input("Enter the dollar value of one Ethereum."))
    print('You entered: ' + str(value_per_coin) + '\n')
    
    total_value = value_per_coin * n_ethereum
    print('Total value in wallet is ' + str(total_value) + ' dollars.')
    
    ser[0] = total_value

if __name__ == "__main__":
    main()
\end{lstlisting}
Attempt the following tasks. In between each task, revert your program back to its original state.
\begin{itemize}
    \item Put an extra space in between \verb|pandas| and \verb|as| in line 1.
    \item Remove \verb|as pd| from line 1.
    \item Remove the opening quote from line 7 inside of the \verb|input| function.
    \item Replace the opening quote on line 7 with a single quote.
    \item Remove the backslash \verb|\| on line 8.
    \item Remove the \verb|[0]| on line 16.
    \item Remove the underscores \verb|_| on line 18.
\end{itemize}
For each task, you will report on what happened and why you think that happened.
\subsection*{Turn In}
You will submit a lab writeup in a Jupyter Notebook. Your lab writeup should contain the following sections.
\begin{itemize}
    \item Introduction: What were you given? What was the purpose of the lab?
    \item Code Description: What does the code (as given to you) do when it is executed?
    \item Tasks: What did each of the items for the list of tasks do? What happened when you made the change as directed? Why do you think that happened?
    \item Conclusion: What did you learn during this lab?
\end{itemize}

\section*{Lab 2: Payroll Calculation}
\subsection*{Description}
An employee is paid at a rate of \$20.68 per hour for the first 40 hours worked in a week. Any hours over that are paid at the overtime rate of one and one half times that. From the worker’s gross pay, 6\% is withheld for social security tax, 14\% is withheld for federal income tax, 6\% is withheld for state income tax, and \$12 per week is withheld for union dues. If the worker has three or more dependents, then an additional \$35 is withheld to cover the extra cost of health insurance beyond what the employer pays.\par
Your job will be to write a program that calculates the values above, given the hours worked and the number of dependents.

\subsection*{Task}
Write a program that will prompt the user for the following information.
\begin{itemize}
    \item The number of hours worked in a week
    \item The number of dependents
\end{itemize}
The program will then output the worker's gross pay, each deduction amount, and the net take-home pay for the week. Write your program so that it allows the calculation to be repeated as often as the user wishes.\par
All decimal point numbers that represent money must be outputted with two digits after the decimal point - no more and no less. For example, print 2.50 instead of 2.5.

\subsection*{Turn In}
You will submit a lab writeup in a Jupyter Notebook. Your lab writeup should contain the following sections.
\begin{itemize}
    \item Introduction: What is the program supposed to do? If you were provided with any code, where did it come from?
    \item Code Description: In your head, break down your program into logical sections. Then, write a subheading for each of your code's sections, include the code (in a code block) what it does, and why you included it.
    \item Issues: Did you experience any issues while writing this lab? If so, what issues did you run into? If not, what are some issues that you could foresee another student making, and how did you avoid these issues? If you were creating this lab, how would you change or improve it?
    \item Completed Program: Include one big code block that contains your program in its entirety.
    \item Test Runs: Provide the entire output from your program when you ran it using the trial data from below.
    \item Conclusion: What did you learn during this lab? How did you apply some of the skills that you've learned to this lab?
\end{itemize}

\subsection*{Trial Data}
\begin{tabular}{lll}
\hline
Trial & Hours & Dependents \\
\hline
1     & 15    & 1          \\
2     & 40    & 4          \\
3     & 53    & 3          \\
4     & 2     & 5          \\
\hline
\end{tabular}

\subsection*{Sample Output}
This sample output is provided to guide you to your solution. You should follow the instructions provided to include all of the functionality that is shown below.
\begin{lstlisting}[style=none]
This program will ask you how many hours you worked, and calculate your
taxes, dues, gross pay, and net pay.

How many hours did you work? 20
How many dependents do you have? 1

Regular hours: 20.00 (at $16.68 an hour)
Overtime hours: 0.00 (at $25.02 an hour)
Total hours: 20.00
Gross pay is $333.60
Social Security tax: $20.02
Federal taxes: $46.70
State taxes: $16.68
Union Dues: $10.00
Total Deductions: $93.40
Net Pay: $240.20.

Would you like to calculate another week's pay? (y or n) y

How many hours did you work? 48
How many dependents do you have? 4

Regular hours: 40.00 (at $16.68 an hour)
Overtime hours: 8.00 (at $25.02 an hour)
Total hours: 48.00
Gross pay is $867.36
Social Security tax: $52.04
Federal taxes: $121.43
State taxes: $43.37
Union Dues: $10.00
Family Health Insurance: $35.00 (additional insurance premiums for your family)
Total Deductions: $261.84
Net Pay: $605.52.

Would you like to calculate another week's pay? (y or n) y

How many hours did you work? 3
How many dependents do you have? 4

Regular hours: 3.00 (at $16.68 an hour)
Overtime hours: 0.00 (at $25.02 an hour)
Total hours: 3.00

Gross pay is $50.04
Social Security tax: $3.00
Federal taxes: $7.01
State taxes: $2.50
Union Dues: $10.00
Family Health Insurance: $35.00 (additional insurance premiums for your family)
Total Deductions: $57.51

Your dues and insurance obligations outstripped your pay by $-7.47.

Would you like to calculate another week's pay? (y or n) n

Thank you for using this program.
\end{lstlisting}

\subsection*{Grading Table}
\begin{tabular}{|l|l|}
\hline
    Requirement & Possible Points \\ \hline
    Correct output on required trial data & 60 \\ \hline
    \makecell[l]{Dollar amounts have the right float value\\(2 decimal places)} & 10 \\ \hline
    Appropriate code formatting, good use of whitespace & 5 \\ \hline
    Meaningful variable names & 5 \\ \hline
    Descriptive comments at the top & 5 \\ \hline
    Descriptive comments to label sections & 5 \\ \hline
    Jupyter Notebook is constructed well and with care & 10 \\ \hline
    \textbf{Total} & \textbf{100} \\ \hline
\end{tabular}

\section*{Lab 3: Bad Programmer!}

\subsection*{Description}
For many programming projects in the real world, you'll be using or working with code that someone else wrote, rather than writing code from scratch. A good programmer will be able to intelligently break down someone else's code, and if you can read someone else's bad code, it'll be a piece of cake to read someone else's good code.

For this lab, you will be writing using terrible programming practices, then you'll be trying to decipher someone else's program, who has also used terrible programming practices.

\subsection*{Task}
For this lab, you will need a partner.

On your own (without your partner), write a program that takes a student's name, year (as an integer), major(s), minor(s), and dormitory. Optional: make the majors and minors multiple choice questions. You can write this in any programming language you'd like as long as it's Python 3.6. Then, print a summary of that student using the format: Your name is [name] and you are a [year]. Your major(s) is in/are [major1] and your minor(s) is in/are [minor1 and minor2]. You live in [dormitory].

Your program should correctly choose whether to use the term "major" or "majors," "minor" or "minors," "is" or "are," and whether you need to use commas or an "and" for multiple majors or minors.

Now, here's the kicker. You know what good programming practices are. Now, break every good programming practice you can without actually breaking valid syntax. That means that your program should properly execute, but it shouldn't be readable by a human. Use bad indentation practices (without breaking Python), bad/nonexistent comments, confoundingly constructed if/then statements and loops, the wrong data structures, obfuscation, complexion and anything else to make your program difficult to read. The only thing you can't do is hide a file. All of your script must fall in one file. If anyone except for you can read your program, you have not done a good job.

Trade programs with your partner. You should've left them with a horrible mess that is virtually unusable. Your partner's job is simple: comment the code that you wrote, and write a summary of the code that you wrote, all without asking you any questions. This summary should contain things like what datatypes are being used for what variables, the logical flow of the program (like where decisions are being made), and how this code might be fixed (though you don't actually have to fix it).

\subsection*{Turn In}
You will submit this program inside of a Jupyter Notebook. You should include both your's and your partner's script (unmodified) inside of code blocks. You should also include your partner's commented and annotated script, as well as your best summary of their code.

\subsection*{Trial Data}
\small{\begin{tabular}{llllll}
\hline
Trial & Name & Year & Majors & Minors & Dormitory \\
\hline
1     & Alice Alexander & Sophomore & Economics & Statistics & Pless Hall \\
\hline
2     & Ben Loch & Junior & \makecell[l]{Env. Science} & & \makecell[l]{McAlester\\Apartment} \\
\hline
3     & Sam Williams & Senior & \makecell[l]{Biology,\\Chemistry} & \makecell[l]{Computer\\Science} & \makecell[l]{Lewis\\Dormitory} \\
\hline
\end{tabular}}

\subsection*{Grading Table}
\begin{tabular}{|l|l|}
\hline
    Requirement & Possible Points \\ \hline
    Correct output on required trial data on your script & 20 \\ \hline
    Inappropriate code formatting,\\bad use of whitespace in your script & 30 \\ \hline
    Poor variable names in your script & 5 \\ \hline
    Poor commenting or no comments,\\other challenges to readability in your script & 5 \\ \hline
    Excellent commenting of partner's code & 10 \\ \hline
    Correct description and interpretation of partner's code & 20 \\ \hline
    Jupyter Notebook is constructed well and with care & 10 \\
    \hline
    \textbf{Total} & \textbf{100} \\ \hline
\end{tabular}

Note: This means that you will \textit{receive} points for writing bad (but syntactically correct) code yourself \textit{and} for properly analyzing and breaking down your partner's bad code.

\section*{Lab 4: Seven-Figure Display}
\subsection*{Description}
You may not know it, but you've seen a seven-figure display before. A seven-figure display uses seven lights to display letters and numbers in relatively simple applications, like microwave ovens and digital clocks.

\includegraphics[width=0.1\textwidth]{img/7Segment5.png}

However, seven figure displays cannot display every character. For example, there is no way to display the letter \verb|g| without it being confused for a \verb|9|. Seven-figure displays are not large enough to display the letter \verb|m|, either. Your goal for this lab is to find the longest word that can be displayed by a seven-segment display.
\subsection*{Task}
Use the \verb|requests| library to download the \verb|words.txt| file from \\\href{https://pythonforscientists.github.io/data/data/words.txt}{https://pythonforscientists.github.io/data/data/words.txt}. This file contains a list of words of varying length, organized alphanumerically. You will use Python to find what the longest word out of this list is that can be displayed on a seven-figure display.\par
Seven-figure displays cannot display the following letters: g, k, m, q, v, w, x, and z.
\subsection*{Turn In}
Turn in a program inside of a Jupyter Notebook. You should include your script, along with the longest word that you found. Include downsides of the approach that you chose and how you might choose to change things if you were to write your program again.
\subsection*{Trial Data}
You should use the \verb|words.txt| file.
\subsection*{Grading Table}
\begin{tabular}{|l|l|}
\hline
Requirement & Possible Points \\ \hline
Correct longest word & 10 \\ \hline
Appropriate whitespace and formatting & 10 \\ \hline
Good variable names & 5 \\ \hline
Good comments and inline documentation & 5 \\ \hline
Downsides identified and explained & 10 \\ \hline
Jupyter Notebook is constructed well & 10 \\ \hline
\hline
\textbf{Total} & \textbf{50} \\ \hline
\end{tabular}

\section*{Lab 5: Wordle}
\subsection*{Description}
Wordle is a simple word guessing game. Your user will be guessing words that are five letters long. They have six chances to get the correct word, otherwise they lose. If they guess a correct letter in any one of the five slots, it is typically marked yellow, and if they guess a correct letter in the correct position, it is typically marked green. Try the Wordle \href{https://www.nytimes.com/games/wordle/index.html}{here}.
\subsection*{Task}
You will program a simplified Wordle, since programming with color is much more difficult. Use the \verb|requests| library to get the \verb|wordle.txt| file from \\\href{https://pythonforscientists.github.io/data/data/wordle.txt}{https://pythonforscientists.github.io/data/data/wordle.txt}. The \verb|wordle.txt| file has 5,757 lower-case words that are five letters long.\par
At random, pick a word, but hide it from the user. Prompt the user to guess what the word is. If they guess a letter in the correct place, display it in its place in a five-letter blank. For example, if they incorrectly guessed \verb|herby| but the third letter is \verb|r|, then display the \verb|r| to confirm the user's guess is correct.
\begin{lstlisting}[style=none]
__r__
\end{lstlisting}
If they guess a letter, but it's not in the correct place, display the letter next to the five-letter blank. For example, if they incorrectly guessed \verb|beret| but third letter is \verb|r| and the fourth letter is \verb|t|, then display the \verb|r| to confirm that letter is correct and the \verb|t| next to the blanks to confirm that the letter is correct, but in the wrong place.
\begin{lstlisting}[style=none]
__r__
In the wrong spot - t

Guess: 
\end{lstlisting}
If the user guesses the secret word in six tries or less, then they win and the game ends.
\begin{lstlisting}[style=none]
Guess: yurts
yurts
You found the word in 4 guesses! Congratulations!
\end{lstlisting}
If the user runs out of guesses, then they lose and the game ends. 
\begin{lstlisting}[style=none]
Guess: stone
_____
You ran out of guesses! The word was yurts. Better luck next time!
\end{lstlisting}
If the user tries to enter a word that's not on the list of words (like 'abcde' or 'xxxxx'), tell the user that the word isn't a valid word, but do not deduct any guesses.
\begin{lstlisting}
Guess: abcde
That's not a word. No guesses used. 6 guesses remain.
\end{lstlisting}
\subsection*{Sample Output}
This sample output is provided to guide you to your solution. You should follow the instructions provided to include all of the functionality that is shown below.
\begin{lstlisting}[style=none]
Guess the word (5 letters, 6 guesses).
Guess: stare
_____
In the wrong spot: a, e

Guess: deans
_____
In the wrong spot: a, e, s

Guess: cleat
__ea_
In the wrong spot: 

Guess: wreak
__ea_
In the wrong spot: r

Guess: smear
smear
You found the word in 5 guesses! Nice!
\end{lstlisting}
\begin{lstlisting}[style=none]
Guess the word (5 letters, 6 guesses).
Guess: stare
s_are
In the wrong spot: 

Guess: spare
s_are
In the wrong spot:

Guess: scare
scare
You found the word in 3 guesses! Nice!
\end{lstlisting}
\begin{lstlisting}[style=none]
Guess the word (5 letters, 6 guesses).
Guess: stare
s_are
In the wrong spot: 

Guess: abcde
That's not a word. No guesses used. 5 guesses remain.

Guess: scare
scare
You found the word in 2 guesses! Nice!
\end{lstlisting}
\begin{lstlisting}[style=none]
Guess the word (5 letters, 6 guesses).
Guess: stare
_____
In the wrong spot: a, e

Guess: deans
_____
In the wrong spot: a, e, s

Guess: cleat
__ea_
In the wrong spot: 

Guess: wreak
__ea_
In the wrong spot: r

Guess: mount
_____
In the wrong spot: m

Guess: swear
s_ear
You ran out of guesses! The word was smear. Better luck next time!
\end{lstlisting}
\subsection*{Bonus}
For bonus points, have the program ask the user whether they'd like to play again. If so, you should keep track of these statistics for the session:
\begin{itemize}
\item What percentage of the time have they gotten the word on their first, second third, fourth, fifth, and sixth try?
\item How many times have they played?
\item How many times have they won?
\item What's their win percentage?
\end{itemize}
After every round, display these statistics to the user. This bonus section is worth 20 extra points, but it must be completed in entirety to be eligible for the bonus. A partially functional bonus is not worth anything.
\subsection*{Turn In}
Turn in your program in a Jupyter Notebook, along with at least four trial runs. Include at least one win and one loss, as well as what happens when an invalid word is entered. Also include a small conclusion on what you would change about your program if you were to write it again. That is, how would you rearrange your program?
\subsection*{Grading Table}
\begin{tabular}{|l|l|}
\hline
Requirement & Possible Points \\ \hline
Correctly functioning program & 20 \\ \hline
Appropriate whitespace and formatting & 20 \\ \hline
Good variable names & 20 \\ \hline
Good comments and inline documentation & 20 \\ \hline
Program refactoring explained & 10 \\ \hline
Jupyter Notebook is constructed well & 10 \\ \hline
\textit{Optional bonus section} & \textit{20} \\\hline
\textbf{Total} & \textbf{100/\textit{120}} \\ \hline
\end{tabular}